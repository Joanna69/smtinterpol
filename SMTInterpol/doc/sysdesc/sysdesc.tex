\documentclass[a4]{article}
\usepackage[english]{babel}
\usepackage{xspace}
\usepackage{hyperref}
\usepackage{bashful}

\newcommand\SI{SMTInterpol\xspace}
\newcommand{\version}{\splice{git describe}}

\title{\SI\\{\large Version \version}}

\author{Jochen Hoenicke, Tanja Schindler\\
  University of Freiburg\\
  \texttt{\{hoenicke,schindle\}@informatik.uni-freiburg.de}}
\date{\today}

\begin{document}
\maketitle
\section*{Description}
\SI is an SMT solver written in Java and available under LGPL v3.  It supports
the quantifier-free combination of the theories of uninterpreted functions,
linear arithmetic over integers and reals, and arrays.  Furthermore it can
produce models, proofs, unsatisfiable cores, and interpolants.  The solver
reads input in SMTLIB format.  It includes parsers for DIMACS, AIGER, and
SMTLIB version 1.2 and 2.5.

The solver uses variants of standard algorithms for CNF
conversion~\cite{DBLP:journals/jsc/PlaistedG86}, congruence
closure~\cite{DBLP:conf/rta/NieuwenhuisO05}, Simplex~\cite{DBLP:conf/cav/DutertreM06} and
branch-and-cut for integer arithmetic~\cite{DBLP:conf/cav/ChristH15,DBLP:conf/cav/DilligDA09}.
The array decision procedure is based on \emph{weak equivalences}~\cite{DBLP:conf/frocos/ChristH15}.
Theory combination is performed based on partial models produced by the theory
solvers~\cite{DBLP:journals/entcs/MouraB08}.

The main focus of \SI is the application track where the incremental
usage of the solver is required.  This track simulates the typical
application of \SI where a user asks multiple queries.  The main focus
of the development team of \SI is the interpolation
engine~\cite{DBLP:journals/jar/ChristH16,conf/ijcar/HoenickeS18}.  This makes it useful as a
backend for software verification tools.  In particular,
Ultimate\-Automatizer, the winner of the SV-COMP 2016 and 2017, uses \SI.

\section*{Competition Version}
The version submitted to the SMT-COMP 2018 is a release based on
version 2.5 with a few recent bug fixes.  This release includes
quantifier-free interpolation for the theory of arrays.

Further information about \SI can be found at
\begin{center}
  \url{http://ultimate.informatik.uni-freiburg.de/smtinterpol/}
\end{center}
The sources are available via GitHub
\begin{center}
  \url{https://github.com/ultimate-pa/smtinterpol}
\end{center}

\section*{Authors}
The code was written by J{\"u}rgen Christ, Jochen Hoenicke, Alexander Nutz, 
Pascal Raiola, and Tanja Schindler.

\section*{Logics, Tracks and Magic Number}

\SI participates in all tracks: the main track, the application track,
and the unsat core track.  It supports the logics QF\_UF, QF\_IDL,
QF\_RDL, QF\_LIA, QF\_LRA, QF\_LIRA, QF\_AX and every combinations of
these logics: QF\_ALIA, QF\_AUFLIA, QF\_AUFLIRA, QF\_UFIDL, QF\_UFLIA,
QF\_UFLRA.

Magic Number: $716\,484\,617$

\bibliography{sysdec}
\bibliographystyle{alpha}
\end{document}
